\documentclass[11pt,psfig]{article}
\usepackage{epsfig}
\usepackage{times}
\usepackage{amssymb}
\usepackage{float}

\newcount\refno\refno=1
\def\ref{\the\refno \global\advance\refno by 1}
\def\ux{\underline{x}}
\def\uw{\underline{w}}
\def\bw{\underline{w}}
\def\ut{\underline{\theta}}
\def\umu{\underline{\mu}} 
\def\bmu{\underline{\mu}} 
\def\be{p_e^*}
\newcount\eqnumber\eqnumber=1
\def\eq{\the \eqnumber \global\advance\eqnumber by 1}
\def\eqs{\eq}
\def\eqn{\eqno(\eq)}

 \pagestyle{empty}
\def\baselinestretch{1.1}
\topmargin1in \headsep0.3in
\topmargin0in \oddsidemargin0in \textwidth6.5in \textheight8.5in
\begin{document}
\setlength{\parskip}{1.2ex plus0.3ex minus 0.3ex}


\thispagestyle{empty} \pagestyle{myheadings} \markright{Homework
1: CS 217 Spring 2015}



\title{CS 217 Homework 1}
\author{Zachary DeStefano, 15247592}
\date{Due Date: April 16, 2015}

\maketitle

\vfill\eject

\newpage

\section*{Problem 1}

Using our assumptions about lenses, we have the following diagram:

\begin{figure}[H]
\centering
\includegraphics[height=4in]{hw1prob1diagram.png}
\caption{Diagram of lens and its focal length}
\end{figure}

The left side of the lens gives us the following equation using similar triangles
\[
\frac{y}{f} = \frac{y+y'}{D'}
\]
The right side of the length gives us the following equation using similar triangles
\[
\frac{y'}{f} = \frac{y+y'}{D}
\]
Adding together the two equations, we get the following
\[
\frac{1}{f} (y + y') = (\frac{1}{D} + \frac{1}{D'})(y + y')
\]
Cancelling the term $y+y'$ we end up with
\[
\frac{1}{f} = \frac{1}{D} + \frac{1}{D'}
\]
This is the thin lens equation


\section{Problem 2}

\subsection{Part 1}

The following diagram illustrates the cross-section showing the horizontal and vertical field of view along with the focal length. The field of view will be $2\theta$

\begin{figure}[H]
\centering
\includegraphics[height=4in]{hw1prob2drawing1.png}
\caption{Diagram of focal length and horizontal/vertical field of view}
\end{figure}

Using trigonometry, it is easily observed that
\[
\theta = arctan(\frac{x}{20f})
\]
\newpage
For the horizontal field of view, $x=640$\\
For the vertical field of view, $x=480$ \\
\\
When $f=50$, the horizontal field of view (in degrees) is as follows: 
\[
2arctan(\frac{640}{20 \cdot 50}) = 65.2385
\]
The vertical field of view (in degrees) is as follows:
\[
2arctan(\frac{480}{20 \cdot 50}) = 51.2820
\]
\\
When $f=100$, the horizontal field of view (in degrees) is as follows: 
\[
2arctan(\frac{640}{20 \cdot 100}) = 35.4893
\]
The vertical field of view (in degrees) is as follows:
\[
2arctan(\frac{480}{20 \cdot 100}) = 26.9915
\]

\end{document}








