\documentclass[11pt,psfig]{article}
\usepackage{epsfig}
\usepackage{times}
\usepackage{amssymb}
\usepackage{float}
\usepackage{listings}
\usepackage{graphicx}
\usepackage{caption}
\usepackage{subcaption}

\newcount\refno\refno=1
\def\ref{\the\refno \global\advance\refno by 1}
\def\ux{\underline{x}}
\def\uw{\underline{w}}
\def\bw{\underline{w}}
\def\ut{\underline{\theta}}
\def\umu{\underline{\mu}} 
\def\bmu{\underline{\mu}} 
\def\be{p_e^*}
\newcount\eqnumber\eqnumber=1
\def\eq{\the \eqnumber \global\advance\eqnumber by 1}
\def\eqs{\eq}
\def\eqn{\eqno(\eq)}

 \pagestyle{empty}
\def\baselinestretch{1.1}
\topmargin1in \headsep0.3in
\topmargin0in \oddsidemargin0in \textwidth6.5in \textheight8.5in
\begin{document}
\setlength{\parskip}{1.2ex plus0.3ex minus 0.3ex}


\thispagestyle{empty} \pagestyle{myheadings} \markright{Crater Lake Reconstruction}

\title{CS 217 Final Project: 3D Reconstruction of Crater Lake}
\author{Zachary DeStefano, 15247592}
\date{Due Date: June 11, 2015}

\maketitle

\vfill\eject

\newpage

\section{Background}

For this project, I attempted a 3D reconstruction of parts of Crater Lake National Park in Oregon.
\begin{figure}[H]
\centering
\includegraphics[width=\columnwidth]{craterLakeWinter.jpg}
\caption{Crater Lake during winter}
\end{figure}
Google Earth offers a 3D view of Crater Lake and my goal was something close to what it showed. There is an island in the middle of the lake called Wizard Island. I decided to focus on that island as well as nearby ridges for my reconstruction. \\
\\
I used pictures taken from different angles in order to do the reconstruction. In order to ensure that I had a wealth of pictures to use, I found high definition video and used a handful of the frames. That way I do not need to worry about using pictures that were taken at different times of the day or year.\\
\\
After looking at multiple high definition videos of Crater Lake, there were two scenes that I decided to attempt to reconstruct. The first one was taken from Merriam Point and it shows Wizard Island as well as nearby peaks. The second one was taken from different points on the surface of the lake and it shows two views of Wizard Island. The first scene had relatively little translational motion thus SFM did not work too well. The second scene had some translational motion thus SFM performed slightly better. 
\begin{figure}[H]
\centering
\includegraphics[width=\columnwidth]{googleEarthView1.png}
\caption{Top View of Crater Lake using Google Earth}
\end{figure}
\begin{figure}[H]
\centering
\includegraphics[width=\columnwidth]{googleEarthView2.png}
\caption{Side View of Crater Lake using Google Earth}
\end{figure}
\begin{figure}[H]
\centering
\includegraphics[width=\columnwidth]{sfmPics1J2/shot4.jpg}
\caption{Merriam Point Scene, Left Camera Shot}
\end{figure}
\begin{figure}[H]
\centering
\includegraphics[width=\columnwidth]{sfmPics1J2/shot26.jpg}
\caption{Merriam Point Scene, Right Camera Shot}
\end{figure}
\begin{figure}[H]
\centering
\includegraphics[width=\columnwidth]{sfmPics3/shot1.jpg}
\caption{Wizard Island Scene, close up shot}
\end{figure}
\begin{figure}[H]
\centering
\includegraphics[width=\columnwidth]{sfmPics3/shot24.jpg}
\caption{Wizard Island Scene, further shot}
\end{figure}

\newpage

\section{Reconstruction of Merriam Point Scene}

The shots taken from Merriam Point had some distinguishable landmarks. There is existing spatial data on these landmarks thus I attempted to do calibration with the 3D data. 

\subsection{3D Calibration Points}

The points I focused on were various peaks as well as recognizable points on the surface of the lake. These points could easily be spotted in a topographic map. Here is the left camera shot marked with the points used:
\begin{figure}[H]
\centering
\includegraphics[width=\columnwidth]{sfmResults1/Photo_withPoints.png}
\caption{Left Camera Shot with Calibration Points marked}
\end{figure}

\newpage

Here are the topographic maps for the points shown in the previous picture. 
\begin{figure}[H]
\centering
\includegraphics[width=\columnwidth]{sfmResults1/TopMap_withPoints_wizardIsland.png}
\caption{Calibration Points near Wizard Island on Topographic Map}
\end{figure}
\begin{figure}[H]
\centering
\includegraphics[width=\columnwidth]{sfmResults1/TopMap_withPoints_peaks.png}
\caption{Calibration Points at the peaks on Topographic Map}
\end{figure}

I used the maps to obtain the $x,y$ coordinates for each of my calibration points. I made the origin the point on both maps where the latitude line is labeled 51 and the longitude line is labeled 47. I used the map scale to obtain a conversion from pixels to feet and my 3D points were in feet.  \\
\\
The z coordinate represented the height in feet. I decided to make the surface of the lake my $z=0$ plane. In order to obtain the z coordinate for my points, I used the following map which showed the exact elevations for each of my points. 
\begin{figure}[H]
\centering
\includegraphics[width=\columnwidth]{sfmResults1/elevationsListed.png}
\caption{Calibration Points with elevations listed}
\end{figure}

\subsection{Triangulation Points using Manual Selection}

After calibrating the left and right images, I decided to try triangulating points by manually selecting them in both images and seeing if a good reconstruction could be calculated. I split the points into 3 groups in order to more easily see how good the reconstruction ended up. Here is the left camera shot with the triangulation points. 
\begin{figure}[H]
\centering
\includegraphics[height=3.5in]{sfmResults1/photoLeft_lakeRidgePoints.png}
\caption{Triangulation Points along the lake}
\end{figure}
\begin{figure}[H]
\centering
\includegraphics[height=3.5in]{sfmResults1/photoLeft_topRidgePoints.png}
\caption{Triangulation Points along the mountain peaks}
\end{figure}
\begin{figure}[H]
\centering
\includegraphics[height=3.5in]{sfmResults1/photoLeft_wizardIslandPoints.png}
\caption{Triangulation Points along Wizard Island}
\end{figure}

Here is the right camera shot with the triangulation points.
\begin{figure}[H]
\centering
\includegraphics[height=3.5in]{sfmResults1/photoRight_lakeRidgePoints.png}
\caption{Triangulation Points along the lake}
\end{figure}
\begin{figure}[H]
\centering
\includegraphics[height=3.5in]{sfmResults1/photoRight_topRidgePoints.png}
\caption{Triangulation Points along the mountain peaks}
\end{figure}
\begin{figure}[H]
\centering
\includegraphics[height=3.5in]{sfmResults1/photoRight_wizardIslandPoints.png}
\caption{Triangulation Points along Wizard Island}
\end{figure}

\newpage

Here is a 3D visualization of the approximated 3D Points. The blue x marks represent the $z=0$ plane. The red circles represent the points on the peak of the mountains in the background. The green circles represent the points along the surface of the lake. The blue circles represent the points along Wizard Island. 
\begin{figure}[H]
\centering
\includegraphics[height=3.5in]{sfmResults1/triangulationAttempt.png}
\caption{Calibration Points with elevations listed}
\end{figure}
\begin{figure}[H]
\centering
\includegraphics[height=3.5in]{sfmResults1/triangulationAttempt2.png}
\caption{Calibration Points with elevations listed}
\end{figure}
\begin{figure}[H]
\centering
\includegraphics[height=3.5in]{sfmResults1/triangulationAttempt3.png}
\caption{Calibration Points with elevations listed}
\end{figure}
\begin{figure}[H]
\centering
\includegraphics[height=3.5in]{sfmResults1/triangulationAttempt4.png}
\caption{Calibration Points with elevations listed}
\end{figure}

As can be observed, the reconstruction proved to be quite poor and while there is some shape, it is not at all what we expect. There are a few likely reasons for this. The biggest challenge is the immense size of what I am trying to image. Because the pictures were taken thousands of feet away from the object, it is likely that the image is relatively flat from the camera's perspective and thus when we move the camera, it does not pick up on much depth. The other major challenge was picking points accurately. Because the object was thousands of feet away, a single pixel error meant the estimate would be off by hundreds of feet. The aggregate of all these errors could easily have a major impact on the camera parameter estimates and final reconstruction. 

\subsection{SIFT Features}

In order to improve my triangulation points, I decided to run SIFT using VLFeat and use the best 100 matches. 


Here was my reconstruction:
\begin{figure}[H]
\centering
\includegraphics[height=3.5in]{sfmResults1/triangulationAttemptSIFT1.png}
\caption{Calibration Points with elevations listed}
\end{figure}
\begin{figure}[H]
\centering
\includegraphics[height=3.5in]{sfmResults1/triangulationAttemptSIFT2.png}
\caption{Calibration Points with elevations listed}
\end{figure}
\begin{figure}[H]
\centering
\includegraphics[height=3.5in]{sfmResults1/triangulationAttemptSIFT3.png}
\caption{Calibration Points with elevations listed}
\end{figure}
\begin{figure}[H]
\centering
\includegraphics[height=3.5in]{sfmResults1/triangulationAttemptSIFT4.png}
\caption{Calibration Points with elevations listed}
\end{figure}

\section{Reconstruction of Wizard Island Scene}


%\begin{figure}[H]
%\centering
%\includegraphics[height=3.5in]{prob1diagram.jpg}
%\caption{Illustration of normal vector N, light vector L, and viewing vector E and their relationships}
%\end{figure}




\end{document}








