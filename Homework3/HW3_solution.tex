\documentclass[11pt,psfig]{article}
\usepackage{epsfig}
\usepackage{times}
\usepackage{amssymb}
\usepackage{float}
\usepackage{listings}
\usepackage{graphicx}
\usepackage{caption}
\usepackage{subcaption}

\newcount\refno\refno=1
\def\ref{\the\refno \global\advance\refno by 1}
\def\ux{\underline{x}}
\def\uw{\underline{w}}
\def\bw{\underline{w}}
\def\ut{\underline{\theta}}
\def\umu{\underline{\mu}} 
\def\bmu{\underline{\mu}} 
\def\be{p_e^*}
\newcount\eqnumber\eqnumber=1
\def\eq{\the \eqnumber \global\advance\eqnumber by 1}
\def\eqs{\eq}
\def\eqn{\eqno(\eq)}

 \pagestyle{empty}
\def\baselinestretch{1.1}
\topmargin1in \headsep0.3in
\topmargin0in \oddsidemargin0in \textwidth6.5in \textheight8.5in
\begin{document}
\setlength{\parskip}{1.2ex plus0.3ex minus 0.3ex}


\thispagestyle{empty} \pagestyle{myheadings} \markright{Homework
2: CS 217 Spring 2015}



\title{CS 217 Homework 3}
\author{Zachary DeStefano, 15247592}
\date{Due Date: May 26, 2015}

\maketitle

\vfill\eject

\newpage

\section*{Problem 1}

If $L$ is the direction of the light source, $N$ is the normal to the surface, we know that the intensity of the light $I$ is given by
\[
I = |L| |N| cos(\alpha)
\]
where $\alpha$ is the angle between $L$ and $N$.\\
That equation is maximized when $cos(\alpha)=1$ so that $\alpha=0$\\
This means that the normal is pointing in the direction of the light source. \\
Thus the light source is in the direction of $(a,b)$ hence finally
\[
L = (\frac{a}{\sqrt{a^2 + b^2}},\frac{b}{\sqrt{a^2 + b^2}})
\]

\end{document}








